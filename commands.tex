\newcommand*\circled[1]{\tikz[baseline=(char.base)]{\node[shape=circle,draw,inner sep=2pt] (char) {#1};}}            

\renewcommand{\baselinestretch}{1.5}
\setlength{\arrayrulewidth}{0,5mm}

\renewcommand{\arraystretch}{1.5}

\newcommand*\mytitle[2]{
\begin{center}
    \begin{Large}
         \textsc{Séquence #1} : #2
    \end{Large}
\end{center}}

\newcommand{\mc}[1]{\begin{multicols}{2}#1\end{multicols}}

\newcommand{\enu}[1]{\begin{enumerate}[label=(\alph*)]#1\end{enumerate}}

\newcommand{\itmz}[1]{\begin{itemize}[label=\textbullet]#1\end{itemize}}

\newcommand{\cntr}[1]{\begin{center}#1\end{center}}

\newcommand{\dtf}{\makebox[\linewidth]{\dotfill}}

%\newcommand{\exercice}{\section{}}

\newcommand{\mkb}[2]{\makebox[#1]{#2}}

\newcommand{\mybox}[1]{\begin{tabular}{|l|}
\hline
#1 \\
\hline
\end{tabular}}

\newcommand{\mkbdtf}[1]{\makebox[#1cm]{\dotfill}}

\newcommand{\mkblw}[1]{\makebox[\linewidth]{#1}}

\newcommand{\myrule}[1]{\rule[2mm]{#1cm}{.1pt}}

\newcommand{\myfigure}[1]{\begin{Figure}
\centering
\includegraphics[width=\linewidth]{images/#1}
\end{Figure}}

\newcommand{\titre}[2]{\cntr{
\begin{LARGE}
\myrule{#2} \textsc{#1} \myrule{#2}
\end{LARGE}}}

\newcommand{\colonnesep}[1]{\setlength{\columnseprule}{ #1 pt}}

\newcommand{\myfig}[2]{\begin{Figure}
\centering
\includegraphics[width=#1]{images/#2}
\end{Figure}}

\newcommand{\tabline}{\\ \hline}

\newcommand{\tableau}[2]{\begin{tabular}{|*{#1}{c|}} \hline #2  \end{tabular}}

\newcommand{\degre}{\ensuremath{^\circ}}

\newcommand{\phm}{\phantom{0}}

\newcommand{\encart}[1]{\framebox[\linewidth]{\rule{0pt}{#1}}}

\newcommand{\nom}{
\textsc{Nom :} \dotfill \\
Prénom : \dotfill \\
Classe : \dotfill}

\newcommand{\diff}[1][99]{ ~~ \ifnum#1=1 $\bigstar\largestar\largestar\largestar$ \else\ifnum#1=2 $ \bigstar\bigstar\largestar\largestar$ \else\ifnum#1=3 $\bigstar\bigstar\bigstar\largestar$ \else\ifnum#1=4  $\bigstar\bigstar\bigstar\bigstar$ \else\ifnum#1>4 \ldots \fi\fi\fi\fi\fi\ }

\newcommand{\calculmental}{
\textbf{Écrire \underline{uniquement les résultats} des calculs affichés au tableau.}

\begin{multicols}{2}
\begin{center}
\begin{enumerate}[label=\protect\circled{\arabic*}]
	\item \makebox[4cm]{\dotfill}
	\item \makebox[4cm]{\dotfill}
	\item \makebox[4cm]{\dotfill}
	\item \makebox[4cm]{\dotfill}
	\item \makebox[4cm]{\dotfill}
	\item \makebox[4cm]{\dotfill}
	\item \makebox[4cm]{\dotfill}
	\item \makebox[4cm]{\dotfill}
	\item \makebox[4cm]{\dotfill}
	\item \makebox[4cm]{\dotfill}
\end{enumerate}
\end{center}
\end{multicols}}

\newcommand{\appreciationnote}[1]{\cntr{
\begin{tabular}{|c|c|}
\hline
\phantom{0000000000000000000000}\textbf{Appréciation}\phantom{0000000000000000000000}&\qquad\qquad\textbf{Note}\qquad\qquad\qquad \\
\hline
& \\
& \resizebox{3cm}{!}{ \ldots/#1 }\\
&\\ \hline
\end{tabular} }
}

\newcommand{\mcs}[2]{\begin{multicols}{#1}#2\end{multicols}}

\newcommand{\bareme}[1]{\hfill 
\textbf{\small \ldots~/ #1 points}}

\newcommand\trou[1]{\small ~~.~ }

\newcommand\interligne[1]{\renewcommand{\baselinestretch}{#1}}

\newcounter{mypage}
\setcounter{mypage}{1}
\newcommand{\comptepage}{\arabic{mypage}\refstepcounter{mypage}}

\newcounter{mycalc}
\setcounter{mycalc}{1}
\newcommand{\comptecalcul}{\Alph{mycalc}\refstepcounter{mycalc}}


\newcommand\bonus{\vspace{0.5cm}{\normalfont\Large\bfseries Exercice (Bonus)}}

\newcounter{myseed}
\setcounter{myseed}{1}
\newcommand{\newseed}{\refstepcounter{myseed}}

\newcounter{myletter}
\setcounter{myletter}{1}
\newcommand{\newletter}{\Alph{myletter}\refstepcounter{myletter}}

\newcommand{\randomletter}{\Alph{\aletter}}

\newcommand{\A}{A}
\newcommand{\B}{B}
\newcommand{\C}{C}

\newcommand{\corrige}{(\textit{Exercice corrigé})}

\pagestyle{fancy}

  
\theoremstyle{plain}

\newtheorem{eexercice}{Exercice}
\newcommand{\exercice}[1]{\begin{eexercice}\phantom{} \par
    {\normalfont #1} \end{eexercice}}

\newtheorem{ttheoreme}{Théorème}
\newcommand{\theoreme}[1]{\begin{ttheoreme} #1 \end{ttheoreme}}

\newtheorem{llemme}{Lemme}
\newcommand{\lemme}[1]{\begin{llemme} #1 \end{llemme}}

\newtheorem{ddemonstration}{Démonstration}
\newcommand{\demonstration}[1]{\begin{ddemonstration} #1 \hfill $\blacksquare$ \end{ddemonstration}}

\newtheorem{ppropriete}{Propriété}
\newcommand{\propriete}[1]{\begin{ppropriete} #1 \end{ppropriete}}

\newtheorem{ddefinition}{Définition}
\newcommand{\definition}[1]{\begin{ddefinition} #1 \end{ddefinition}}

\newtheorem{eexemple}{Exemple}
\newcommand{\exemple}[1]{\begin{eexemple} #1 \end{eexemple}}

\newtheorem{pprobleme}{Problème}
\newcommand{\probleme}[1]{\begin{pprobleme} #1 \end{pprobleme}}

\newtheorem*{rremarque}{Remarque}
\newcommand{\remarque}[1]{\begin{rremarque} #1 \end{rremarque}}

\newtheorem*{nnotation}{Notation}
\newcommand{\notation}[1]{\begin{nnotation} #1 \end{nnotation}}

\newtheorem*{vvocabulaire}{Vocabulaire}
\newcommand{\vocabulaire}[1]{\begin{vvocabulaire} #1 \end{vvocabulaire}}

\newtheorem*{rrappel}{Rappel}
\newcommand{\rappel}[1]{\begin{rrappel} #1 \end{rrappel}}

% Begin expl3 code
\ExplSyntaxOn
% Define variables for year handling
\int_new:N \l_my_year_int
\int_new:N \l_my_previous_year_int
\int_new:N \l_my_next_year_int
\int_new:N \l_my_month_int

% Set the current year and month
\int_set:Nn \l_my_year_int {\the\year}
\int_set:Nn \l_my_month_int {\the\month}

% Adjust years based on the month
\int_set:Nn \l_my_previous_year_int {\int_eval:n {\l_my_year_int - 1}}
\int_set:Nn \l_my_next_year_int {\int_eval:n {\l_my_year_int + 1}}

% Define the academic year command
\newcommand{\academicyear}{
  \int_compare:nNnTF {\l_my_month_int} < 8
    { \int_use:N \l_my_previous_year_int - \int_use:N \l_my_year_int }
    { \int_use:N \l_my_year_int - \int_use:N \l_my_next_year_int }
}

\ExplSyntaxOff

\renewcommand{\headrulewidth}{.5pt}
\fancyhead[L]{\textbf{\academicyear}} 
%\fancyhead[C]{}
\fancyhead[R]{\textbf{CSMH}}

%\fancyfoot[L]{}
\fancyfoot[C]{\textbf{(\thepage)}} 
% \fancyfoot[R]{}




